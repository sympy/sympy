%&latex
\documentclass{article}
\usepackage[dvips]{graphicx,epsfig,rotating,color}
\usepackage{latexsym,epic,eepic,pstricks}
\usepackage{graphicx}
\usepackage{longtable}
\usepackage{hyperref}
\usepackage{verbatim}
\usepackage{tabularx}
\usepackage{bm}
\newcommand{\lp}{\left (}
\newcommand{\rp}{\right )}
\newcommand{\lbrk}{\left [}
\newcommand{\rbrk}{\right ]}
\newcommand{\be}{\begin{equation}}
\newcommand{\ee}{\end{equation}}
\newcommand{\ds}{\displaystyle}
\newcommand{\bfrac}[2]{{\ds\frac{#1}{#2}}}
\newcommand{\lbrc}{\left \{}
\newcommand{\rbrc}{\right \}}
\newcommand{\half}{\bfrac{1}{2}}
\newcommand{\W}{\wedge}
\newcommand{\abs}[1]{\left | {#1} \right |}
\newcommand{\Fof}[1]{F\lp{#1}\rp}
\newcommand{\ebar}{\bar{e}}
\newcommand{\nbar}{\bar{n}}
\newcommand{\proj}[2]{\left <{#1}\right >_{#2}}





    

\newcommand{\rotxt}[1]{\begin{sideways}{#1}\end{sideways}}                                                 
 
\setlength{\parindent}{0pt}
\setlength{\parskip}{1.5ex plus 0.5ex minus 0.5ex}

\begin{document}

\title{{\bf DESIGN OF A PYTHON MODULE FOR SYMBOLIC GEOMETRIC ALGEBRA CALCULATIONS}}

\author{Alan Bromborsky\\
abrombo@verizon.net\\
12435 Kemp Mill Road\\
Silver Spring, MD 20902}
\maketitle

\begin{abstract}
A python module (GAsympy.py) has been developed for coordinate free calculations
using the operations (geometric, outer, and inner products etc.) of geometric
algebra.  The operations can be defined using a completely arbitrary pseudometric
defined by the inner products of a set of arbitrary vectors or the pseudometric 
can be restricted to enforce orthogonality and signature constraints on the set
of vectors.  The module requires the numpy and the sympy modules.  A simple 
calculator program is included for those who do not wish to program in python.
\end{abstract}

\section{Introduction}
Several software packages for numerical geometric algebra calculations are available from Doran-Lazenby
group and the Dorst group. Symbolic packages for Clifford algebra using orthongonal bases such as
$e_{i}e_{j}+e_{j}e_{i} = 2\eta_{ij}$, where $\eta_{ij}$ is a numeric array are available from the 
Doran-Lazenby group. The symbolic algebra module, GAsympy.py, developed for python does note depend on
an orthogonal basis representation, but rather is generated from a set of $n$ arbitrary symbolic vectors, 
$a_{1},a_{2},\dots,a_{n}$ and a symbolic pseudo-metric tensor $g_{ij} = a_{i}\cdot a_{j}$.

In order not to reinvent the wheel all scalar symbolic algebra is handled by the python module (library) 
sympy. {\bf The basic classes used from the sympy module 
are the creation of symbolic scalar symbols and numerical symbols for exact rational arithmetic:}
\newpage
{\bf Examples of sympy class usage}

\begin{verbatim}
import sympy
x = sympy.Symbol('x')
half = sympy.Rational(4,8)
print x,half
	x,1/2
\end{verbatim}
as you can see in the example rational numbers are simplified. Also, if we had used {\tt 'ab'} for the 
argument of {\tt sympy.Symbol} then {\tt print x} would have returned {\tt ab}.

The basic geometic algebra operations will be implemented in python by defining a multivector 
class, MV, and overloading the class operators and defining class functions shown in Table~\ref{table1}.
\begin{table}[h!]
\begin{tabular}{cl}
	{\tt +} & sum of multivectors or multivector and scalar \\
	{\tt -} & difference of multivectors  or multivector and scalar \\
	{\tt *} & geometric product or multiplication by scalar \\
	{\tt \verb!^!} & outer product of multivectors \\
	{\tt |} & inner product of multivectors \\	
	{\tt rev()} & reverse of multivector \\
	{\tt even()} & even part of multivector \\
	{\tt odd()} & odd part of multivector \\
	{\tt project(r)} & grade {\tt r} part of multivector \\
	{\tt X(ig,ib)} & \parbox[t]{3.5in}{For a multivector {\tt X} the call operator gives the symbolic 
					  coefficient of the {\tt ib} base of the {\tt ig} grade of the 
                      multivector.  Because of the defaults in the function call {\tt X()} returns 
                      the scalar part of {\tt X}.}
\end{tabular}
\caption{Multivector operations for symbolicGA.py.}\label{table1}
\end{table}
Note that the operator order precedence is determined by python and is not neccessarily that used by 
geometric algebra. Always use parenthesis in python expressions containing {\tt $\W$} and/or {\tt |}.

Complete documentation of GAsympy is obtained by running {\tt pydoc GAsympy} or {\tt pydoc -w GAsympy} 
(for html document) in the directory containing GAsympy.

If the reader is not familiar with python the recommended reference is {\bf Learning Python: 3$^{rd}$ 
edition} by Mark Lutz.

The GAsympy module was implemented in python 2.5.1 on a linux machine running Kubuntu.  The only 
required packages (in addition to the standard python installation) are numpy and sympy.

\section{Vector Basis and Metric}

The two structures that define the {\tt MV} (multivector) class are the symbolic basis vectors and the 
symbolic pseudometric.  The symbolic basis vectors are input as a string with the symbol
name separated by spaces.  For example if we are calculating the geometric algebra of a system with three
vectors that we wish to denote as {\tt a0}, {\tt a1}, and {\tt a2} we would define the string variable:

{\tt basis = 'a0 a1 a2'}

that would be input into the multivector setup function.  The next step would be to define the symbolic 
pseudometric for the geometric algebra of the basis we have defined. The default basis is the most general
and is the matrix of the following symbols

\be\label{eq1}
g = \lbrk
\begin{array}{ccc}
	{\tt a0**2} & {\tt (a0.a1)}  & {\tt (a0.a2)} \\ 
	{\tt (a0.a1)} & {\tt a1**2}  & {\tt (a1.a2)} \\
	{\tt (a0.a2)} & {\tt (a1.a2)} & {\tt a2**2} \\
\end{array}
\rbrk
\ee

where each of the $g_{ij}$ is a symbol representing all of the dot products of the basis vectors. Note that the symbols are named so that $g_{ij} = g_{ji}$ since for the symbol function ${\tt (a0.a1)} \ne {\tt (a1.a0)}$.

Note that the strings shown in equation~\ref{eq1} are only used when the values of $g_{ij}$ are output (printed).  
In the {\tt GAsympy} module (library) the $g_{ij}$ symbols are stored in a static member list of the multivector class
{\tt MV} as the double list {\tt MV.metric} ($g_{ij} = $ {\tt MV.metric[i][j]}).

The default definition of $g$ can be overwritten by specifying a string that will define $g$. As an example 
consider a symbolic representation for conformal geometry. Define for a basis    

{\tt basis = 'a0 a1 a2 n nbar'}

and for a metric

\verb!metric = '# # # 0 0, # # # 0 0, # # # 0 0, 0 0 0 0 2, 0 0 0 2 0'!

which is processed by setup to yield

\be
g = \lbrk
\begin{array}{ccccc}
	{\tt a0**2} & {\tt (a0.a1)}  & {\tt (a0.a2)} & 0 & 0\\ 
	{\tt (a0.a1)} & {\tt a1**2}  & {\tt (a1.a2)} & 0 & 0\\
	{\tt (a0.a2)} & {\tt (a1.a2)} & {\tt a2**2} & 0 & 0 \\
	0 & 0 & 0 & 0 & 2 \\
	0 & 0 & 0 & 2 & 0
\end{array}
\rbrk
\ee 

Here we have specified that {\tt n} and {\tt nbar} are orthonal to all the {\tt a}'s,
${\tt n**2} = {\tt nbar**2} = 0$, and ${\tt (n.nbar)} = 2$. Using {\tt \#} in the metric
definition string just tells the program to use the default symbol for that value.  The multivector
algebra is then initialized with the function call

{\tt MV.setup(basis,metric)}

At this time multivector representations of the basis local to the program are instantiated.  For our
first example that means that the symbolic multivectors named {\tt a0}, {\tt a1}, and {\tt a2} are
created and made available to the programmer for future calculations.

In addition to the basis vectors the $g_{ij}$ are also made available to the programer with the following
convention. If {\tt a0} and {\tt a1} are basis vectors, then their dot products are denoted by 
{\tt a0sq}, {\tt a2sq}, and {\tt a0dota1} for use as python program varibles. If you print 
{\tt a0sq} the output would be {\tt a0**2} and the output for {\tt a0dota1} would be {\tt (a0.a1)} as
shown in equation~\ref{eq1}.  If the default value are overridden the new values are output by print.  For
examle if $g_{00} = 0$ then "{\tt print a0sq}" would output "0."


More generally, if {\tt metric} is not a string, but a list of lists, it is assumed that each element
of {\tt metric} is symbolic variable so that the $g_{ij}$ could be defined as symbolic functions as well
as variables. For example instead of letting $g_{01} = {\tt (a0.a1)}$ we could have 
$g_{01} = {\tt cos(theta)}$ where we use a symbolic $\cos$ function.

\section{Representation and Reduction of Multivector Bases}

{\bf Python Note: In python a list of objects is denoted with square brackets {\tt []}. For example a list
of the integers 1, 2, and 3 would be written in python code as {\tt [1,2,3]}. An object in a list can itself
be a list (nested lists).  Objects in a list containing $n$-objects are indexed $0,1,\dots,n-1$ where if
{\tt B} is the list, {\tt B[0]} is the first object in the list and {\tt B[n-1]} is the last object in the list.}

In our symbolic geometric algebra we assume that all multivectors of interest to us can be obtained from the
symbolic bases vectors we have input, via the different operations available to geometric algebra. The first
problem we have is representing the general multivector in terms terms of the basis vectors.  To do this we 
form the ordered geometric products of the basis vectors and develop an internal representation of these 
products in terms of python classes.  The ordered geometric products are all multivectors of the form
$a_{i_{1}}a_{i_{2}}\dots a_{i_{r}}$ where $i_{1}<i_{2}<\dots <i_{r}$ and $r \le n$. We call these
multivectors bases and represent them internally with the list of integers 
$\lbrk i_{1},i_{2},\dots,i_{r}\rbrk$. The bases are labeled, for the purpose of output display, with strings
that are concatenations of the strings representing the basis vectors.  So that in our example 
{\tt [1,2]} would be labeled with the string {\tt 'a1a2'} and represents the geometric product 
{\tt a1*a2}. Thus the list {\tt [0,1,2]} represents {\tt a0*a1*a2}.For our example the complete set
 of bases and labels
are shown in Table~\ref{table2}\footnote{The empty list, {\tt[]}, represents the scalar 1.}
\begin{table}[h]
\begin{verbatim}
MV.basislabel = ['1', ['a0', 'a1', 'a2'], ['a0a1', 'a0a2', 'a1a2'], 
                 ['a0a1a2']]
MV.basis      = [[], [[0], [1], [2]], [[0, 1], [0, 2], [1, 2]], 
                 [[0, 1, 2]]]
\end{verbatim}
\caption{Multivector basis labels and internal basis representation.}\label{table2}
\end{table}
Since there are $2^{n}$ bases and the number of bases with equal list lengths is the same as for the 
grade decomposition of a dimension $n$ geometric algebra we will call the collections of bases of equal length
{\bf psuedogrades}.

The critical operation in setting up the geometric algebra module is reducing the geomertric product of any two bases to
a linear combination of bases so that we can calculate a multiplication table for the bases.  
First we represent the product as the concatenation of two base lists.  For
example {\tt a1a2*a0a1} is represented by the list {\tt [1,2]+[0,1] = [1,2,0,1]} (In python the "+" operator for lists concatenates the lists). The representation of the product is reduced via two operations, contraction and
revision. The state of the reduction is saved in two lists of equal length.  The first list contains symbolic
scale factors (symbol or numeric types) for the corresponding interger list representing the product of 
bases.  If we wish to reduce $\lbrk i_{1},\dots,i_{r}\rbrk$ the starting point is the coefficient list 
$C = \lbrk 1 \rbrk$ and the bases list $B = \lbrk \lbrk i_{1},\dots,i_{r}\rbrk \rbrk$.  We now operate 
on each element of the lists as follows:

\begin{center}
\begin{tabularx}{4.5in}{cXlX}\hline
	{\bf contraction} &  \parbox{3.5in}{\vspace{0.1in}Consider a basis list $B$ with element
						 $B[j] = \lbrk i_{1},\dots,i_{l},i_{l+1},\dots,i_{s}\rbrk$ where 
					     $i_{l} = i_{l+1}$. Then the product of the $l$ and $l+1$ terms result in a scalar and $B[j]$ is
						replaced by the new list
						 representation $\lbrk i_{1},\dots,i_{l-1},i_{l+2},\dots,i_{r}\rbrk$ which is of 
						 psuedo grade $r-2$ and $C[j]$ is replaced by the symbol $g_{i_{l}i_{l}}C[j]$.\vspace{0.1in}} \\
						 \hline
\end{tabularx}
\begin{tabularx}{4.5in}{cXlX}\hline 
    {\bf revision}    & \parbox{3.5in}{\vspace{0.1in}Consider a basis list $B$ with element
						 $B[j] = \lbrk i_{1},\dots,i_{l},i_{l+1},\dots,i_{s}\rbrk$ where 
					     $i_{l} > i_{l+1}$. Then the $l$ and $l+1$ elements must be reversed to be
						 put in normal order,
						 but we have $a_{i_{l}}a_{i_{l+1}} = 2g_{i_{l}i_{l+1}}-
						 a_{i_{l+1}}a_{i_{l}}$ (From the geometric 
						 algebra definition of the dot product of two vectors). Thus
						 we append the list representing the reduced element ,
						 $\lbrk i_{1},\dots,i_{l-1},i_{l+2},\dots,i_{s}\rbrk$, to the pseudo bases list, $B$, and
						 append $2g_{i{l}i_{l+1}}C[j]$ to the coefficients list, then 
						 we replace $B[j]$ with $\lbrk i_{1},\dots,i_{l+1},i_{l},\dots,i_{s}\rbrk$ 
					     and $C[j]$ with $-C[j]$. Both lists are increased by one element if 
						 $g_{i_{l}i_{l+1}} \ne 0$.\vspace{0.1in}} \\ \hline
\end{tabularx}
\end{center}

These processes are repeated untill every bases list in $B$ is in normal (ascending) order with no repeated elements.  Then the coefficents of equivalent bases are summed and the bases sorted according to psuedograde
and ascending order.  We now have a way of calculating the geometric product of any two bases as a symbolic linear combination of all the bases with the coefficients determined by $g$. The base multiplication table
for our simple example of three vectors is given by (the coefficient of each psuedo base is enclosed with \{\} for clarity):

{\small
\begin{verbatim}
(1)(1) = 1
(1)(a0) = a0
(1)(a1) = a1
(1)(a2) = a2
(1)(a0a1) = a0a1
(1)(a0a2) = a0a2
(1)(a1a2) = a1a2
(1)(a0a1a2) = a0a1a2

(a0)(1) = a0
(a0)(a0) = {a0**2}1
(a0)(a1) = a0a1
(a0)(a2) = a0a2
(a0)(a0a1) = {a0**2}a1
(a0)(a0a2) = {a0**2}a2
(a0)(a1a2) = a0a1a2
(a0)(a0a1a2) = {a0**2}a1a2

(a1)(1) = a1
(a1)(a0) = {2*(a0.a1)}1-a0a1
(a1)(a1) = {a1**2}1
(a1)(a2) = a1a2
(a1)(a0a1) = {-a1**2}a0+{2*(a0.a1)}a1
(a1)(a0a2) = {2*(a0.a1)}a2-a0a1a2
(a1)(a1a2) = {a1**2}a2
(a1)(a0a1a2) = {-a1**2}a0a2+{2*(a0.a1)}a1a2

(a2)(1) = a2
(a2)(a0) = {2*(a0.a2)}1-a0a2
(a2)(a1) = {2*(a1.a2)}1-a1a2
(a2)(a2) = {a2**2}1
(a2)(a0a1) = {-2*(a1.a2)}a0+{2*(a0.a2)}a1+a0a1a2
(a2)(a0a2) = {-a2**2}a0+{2*(a0.a2)}a2
(a2)(a1a2) = {-a2**2}a1+{2*(a1.a2)}a2
(a2)(a0a1a2) = {a2**2}a0a1+{-2*(a1.a2)}a0a2+{2*(a0.a2)}a1a2

(a0a1)(1) = a0a1
(a0a1)(a0) = {2*(a0.a1)}a0+{-a0**2}a1
(a0a1)(a1) = {a1**2}a0
(a0a1)(a2) = a0a1a2
(a0a1)(a0a1) = {-a0**2*a1**2}1+{2*(a0.a1)}a0a1
(a0a1)(a0a2) = {2*(a0.a1)}a0a2+{-a0**2}a1a2
(a0a1)(a1a2) = {a1**2}a0a2
(a0a1)(a0a1a2) = {-a0**2*a1**2}a2+{2*(a0.a1)}a0a1a2

(a0a2)(1) = a0a2
(a0a2)(a0) = {2*(a0.a2)}a0+{-a0**2}a2
(a0a2)(a1) = {2*(a1.a2)}a0-a0a1a2
(a0a2)(a2) = {a2**2}a0
(a0a2)(a0a1) = {-2*a0**2*(a1.a2)}1+{2*(a0.a2)}a0a1+{a0**2}a1a2
(a0a2)(a0a2) = {-a0**2*a2**2}1+{2*(a0.a2)}a0a2
(a0a2)(a1a2) = {-a2**2}a0a1+{2*(a1.a2)}a0a2
(a0a2)(a0a1a2) = {a0**2*a2**2}a1+{-2*a0**2*(a1.a2)}a2+{2*(a0.a2)}a0a1a2

(a1a2)(1) = a1a2
(a1a2)(a0) = {2*(a0.a2)}a1+{-2*(a0.a1)}a2+a0a1a2
(a1a2)(a1) = {2*(a1.a2)}a1+{-a1**2}a2
(a1a2)(a2) = {a2**2}a1
(a1a2)(a0a1) = {2*a1**2*(a0.a2)-4*(a0.a1)*(a1.a2)}1+{2*(a1.a2)}a0a1+{-a1**2}a0a2
              +{2*(a0.a1)}a1a2
(a1a2)(a0a2) = {-2*a2**2*(a0.a1)}1+{a2**2}a0a1+{2*(a0.a2)}a1a2
(a1a2)(a1a2) = {-a1**2*a2**2}1+{2*(a1.a2)}a1a2
(a1a2)(a0a1a2) = {-a1**2*a2**2}a0+{2*a2**2*(a0.a1)}a1+{2*a1**2*(a0.a2)
                -4*(a0.a1)*(a1.a2)}a2+{2*(a1.a2)}a0a1a2

(a0a1a2)(1) = a0a1a2
(a0a1a2)(a0) = {2*(a0.a2)}a0a1+{-2*(a0.a1)}a0a2+{a0**2}a1a2
(a0a1a2)(a1) = {2*(a1.a2)}a0a1+{-a1**2}a0a2
(a0a1a2)(a2) = {a2**2}a0a1
(a0a1a2)(a0a1) = {2*a1**2*(a0.a2)-4*(a0.a1)*(a1.a2)}a0+{2*a0**2*(a1.a2)}a1
                +{-a0**2*a1**2}a2+{2*(a0.a1)}a0a1a2
(a0a1a2)(a0a2) = {-2*a2**2*(a0.a1)}a0+{a0**2*a2**2}a1+{2*(a0.a2)}a0a1a2
(a0a1a2)(a1a2) = {-a1**2*a2**2}a0+{2*(a1.a2)}a0a1a2
(a0a1a2)(a0a1a2) = {-a0**2*a1**2*a2**2}1+{2*a2**2*(a0.a1)}a0a1+{2*a1**2*(a0.a2)
                  -4*(a0.a1)*(a1.a2)}a0a2+{2*a0**2*(a1.a2)}a1a2
\end{verbatim}
}

\section{Base Representation of Multivectors}

{\bf Python Note: In GAsympy symbolic multivectors are instanciated as the python class {\tt MV}. This means that 
associated with each instance of a particular multivector {\tt A} is a set of data which defines that particular
multivector and a set of operations (functions such as projection and reversal and operations such as addition, 
subtraction, geometric product, innner product, and outer product) defined by the class. For a particular {\tt A} most of
the data related to {\tt A} is stored in the list of arrays {\tt A.mv}. In addition there is a flag, {\tt A.bladeflg},
that would indicate if {\tt A} is currently a base or blade representation (see section 5) and data structures 
for any other required information.}

In terms of the bases defined an arbitrary multivector can be represented as a list of arrays 
(we use the numpy python module to implement arrays).  
If we have $n$ basis vectors we initialize the list {\tt self.mv = [0,0,...,0]} with $n+1$ zeros.  Each zero is a
placeholder for an array of python objects (in this case the objects will be sympy 
symbol objects). If {\tt self.mv[r] = numpy.array([list of symbol objects])} each entry in the {\tt numpy.array}
will be a coefficient of the corresponding psuedo base. {\tt self.mv[r] = 0} indicates that the coefficients of every
base of psuedo grade $r$ are 0.  The length of the array {\tt self.mv[r]} is $n \choose r$ the
binomial coefficient. For example the psuedo basis vector {\tt a1} would be represented as a multivector by the list:

{\tt a1.mv = [0,numpy.array([numeric(0),numeric(1),numeric(0)]),0,0]}

and {\tt a0a1a2} by:

{\tt a0a1a2.mv = [0,0,0,numpy.array([numeric(1)])]}

The array is stuffed with sympy numeric objects instead of python integers so that we can perform symbolically manipulate
sympy expressions that consist of scalar algebraic symbols and exact rational numbers which sympy can also represent.

The {\tt numpy.array} is used because operations of addition, substraction, and multiplication by an object are
defined for the array if they are defined for the objects making up the array, which they are by sympy. 
We call this representation a base type because the {\tt r} index is not a grade index since the bases we are using are not blades. In a blade representation the structure would be identical, but the bases would be replaced by 
blades and {\tt self.mv[r]} would represent the {\tt r} grade components of the multivector.  The first use of the
base representation is to store the results of the multiplication tabel for the bases in the class variable {\tt MV.mtabel}.  This variable is a group of nested lists so that the geometric product of the {\tt igrade} and 
{\tt ibase} with the {\tt jgrade} and {\tt jbase} is {\tt MV.mtabel[igrade][ibase][jgrade][jbase]}.  We can then use this table to calculate the geometric product of any two multivectors.

\section{Blade Representation of Multivectors}

Since we can now calculate the symbolic geometric product of any two multivectors we can also calculate the
blades corresponding to the product of the symbolic basis vectors using the formula
\be
	A_{r}\W b = \half\lp A_{r}b-\lp -1 \rp^{r}bA_{r} \rp,
\ee
where $A_{r}$ is a multivector of grade $r$ and $b$ is a vector.  For our example basis the result 
is shown in Table~\ref{table3}.
\begin{table}[h]
\begin{verbatim}
1 = 1
a0 = a0
a1 = a1
a2 = a2
a0^a1 = {-(a0.a1)}1+a0a1
a0^a2 = {-(a0.a2)}1+a0a2
a1^a2 = {-(a1.a2)}1+a1a2
a0^a1^a2 = {-(a1.a2)}a0+{(a0.a2)}a1+{-(a0.a1)}a2+a0a1a2
\end{verbatim}
\caption{Bases blades in terms of bases.}\label{table3}
\end{table}
The important thing to notice about Table~\ref{table3} is that it is a triagonal (lower triangular) system of equations so that
using a simple back substitution algorithym we can solve for the psuedo bases in terms of the blades giving Table~\ref{table4}.  
\begin{table}[h]
\begin{verbatim}
1 = 1
a0 = a0
a1 = a1
a2 = a2
a0a1 = {(a0.a1)}1+a0^a1
a0a2 = {(a0.a2)}1+a0^a2
a1a2 = {(a1.a2)}1+a1^a2
a0a1a2 = {(a1.a2)}a0+{-(a0.a2)}a1+{(a0.a1)}a2+a0^a1^a2
\end{verbatim}
\caption{Bases in terms of bases blades.}\label{table4}
\end{table}
Using Table~\ref{table4} and simple substitution we can convert from a base multivector representation to a
blade representation.  Likewise, using Table~\ref{table3} we can convert from 
blades to bases.

Using the blade representation it becomes simple to program functions that will calculate the grade projection,
reverse, even, and odd multivector functions.

Note that in the multivector class {\tt MV} there is a class variable for each instantiation, {\tt self.bladeflg}, that is set to zero
for a base representation and 1 for a blade representation.  One needs to keep track of which representation is
in use since various multivector operations require conversion from one representation to the other.

\section{Outer and Inner Product}

{\bf Geometric Algebra Note: In geometric algebra any general multivector $A$ can be decomposed into pure grade 
multivectors (a linear combination of blades of all the same order) so that in a $n$-dimensional vector space
\be
A = \sum_{r = 0}^{n}A_{r}
\ee
The geometric product of two pure grade multivectors $A_{r}$ and $B_{s}$ has the form
\be
A_{r}B_{s} = \proj{A_{r}B_{s}}{\abs{r-s}}+\proj{A_{r}B_{s}}{\abs{r-s}+2}+\cdots+\proj{A_{r}B_{s}}{r+s}
\ee
where $\proj{}{t}$ projects the $t$ grade components of the multivector argument.  The  
inner and outer products of $A_{r}$ and $B_{s}$ are then defined to be
\be
A_{r}\cdot B_{s} = \proj{A_{r}B_{s}}{\abs{r-s}}
\ee
\be
A_{r}\wedge B_{s} = \proj{A_{r}B_{s}}{r+s}
\ee
and
\be
A\cdot B = \sum_{r,s}A_{r}\cdot B_{s}
\ee
\be
A\wedge B = \sum_{r,s}A_{r}\wedge B_{s}
\ee
}

The {\tt MV} class function for the outer product of the multivectors {\tt mv1} and {\tt mv2} is

\begin{verbatim}
   def outer_product(mv1,mv2):
        product = MV()
        product.bladeflg = 1
        mv1.convert_to_blades()
        mv2.convert_to_blades()
        for igrade1 in MV.n1rg:
            if not isint(mv1.mv[igrade1]):
                pg1 = mv1.project(igrade1)
                for igrade2 in MV.n1rg:
                    igrade = igrade1+igrade2
                    if igrade <= MV.n:
                        if not isint(mv2.mv[igrade2]):
                            pg2 = mv2.project(igrade2)
                            pg1pg2 = pg1*pg2
                            product.add_in_place(pg1pg2.project(igrade))
        return(product)
    outer_product = staticmethod(outer_product) 
\end{verbatim}

In the 
{\tt MV} class we have overloaded the {\tt \verb!^!} operator so that instead of calling the function we
can write {\tt mv1\verb!^!mv2}.  Due to the precedence rules for python we should {\bf always} enclose outer
and inner products in parenthesis.  The steps for calculating the outer product are:
\begin{enumerate}
\item Convert {\tt mv1} and {\tt mv2} to blade representation if they are not already in that form.
\item Project and loop through each grade {\tt mv1.mv[i1]} and {\tt mv2.mv[i2]}.
\item Calculate the geometric product {\tt pg1*pg2}.
\item Project the {\tt i1+i2} grade from {\tt pg1*pg2}.
\item Accumulate the results for each pair of grades in the imput multivectors.
\end{enumerate}
For the inner product of the multivectors {\tt mv1} and {\tt mv2} the {\tt MV} class function is 

\begin{verbatim}
    def inner_product(mv1,mv2):
        product = MV()
        product.bladeflg = 1
        mv1.convert_to_blades()
        mv2.convert_to_blades()
        for igrade1 in range(1,MV.n1):
            if not isint(mv1.mv[igrade1]):
                pg1 = mv1.project(igrade1)
                for igrade2 in range(1,MV.n1):
                    igrade = abs(igrade1-igrade2)
                    if not isint(mv2.mv[igrade2]):
                        pg2 = mv2.project(igrade2)
                        pg1pg2 = pg1*pg2
                        product.add_in_place(pg1pg2.project(igrade))
        return(product)
    inner_product = staticmethod(inner_product)  
\end{verbatim}

In the {\tt MV} class we have overloaded the {\tt |} operator so that instead 
of calling the function we can write {\tt mv1|mv2}. The inner product is calculated the same way 
as the outer product except that in step~4, {\tt i1+i2} is replaced by {\tt abs(i1-i2)}.

\section{Examples of GAsympy in Action}

We now give three examples of symbolicGA in use (all are from {\bf Geometric Algebra for Physicists}
by Doran and Lazenby). The complete python code for all three examples is given in the file
{\tt test\_symbolicGA.py} and the results in {\tt test\_symbolicGA.out}.

\subsection{Derive Non-Euclidian Distance}

We shall derive the formula for caculating the distance in hyperbolic space
from chapter~10 of {\bf Geometric Algebra for Physicists} by Doran and Lazenby. The equations we must 
solve (pages 373-374) are
\be
 B = \lp X\W Y\W e\rp e = Le, \mbox{ where }X^{2} = Y^{2} = 0 \mbox{ and }e^{2} = 1
\ee
\be
	\hat{B} = \bfrac{B}{\sqrt{B^{2}}}
\ee
\be\label{eq6}
	Y = e^{\bfrac{\alpha\hat{B}}{2}}Xe^{-\bfrac{\alpha\hat{B}}{2}}
\ee
To solve equation~\ref{eq6} note that $Y\W Y = 0$ so that equation~\ref{eq6} is satisfied if,
\be\label{eq7}
    \lp e^{\bfrac{\alpha\hat{B}}{2}}Xe^{-\bfrac{\alpha\hat{B}}{2}}\rp \cdot Y = 0
\ee
and we use
\be
	e^{\bfrac{\alpha\hat{B}}{2}} = \cosh\lp\frac{\alpha}{2}\rp+\sinh\lp\frac{\alpha}{2}\rp\hat{B}
\ee
to solve for $\alpha$ as a function of $X$ and $Y$.

The python code used to solve equation~\ref{eq7} is shown below.  Note the comments (all lines
between """'s) that explain the inputs and outputs.

\begin{verbatim}
	print 'Example: non-euclidian distance calculation'

	metric = '0 # #,'+ \
			 '# 0 #,'+ \
			 '# # 1,'

	MV.setup('X Y e',metric,debug=0)
	MV.set_str_format(1)

	"""
	X and Y are conformal mappings of two vectors on the Poincare disk
	"""

	L = X^Y^e

	"""
	B is the bivector generator of translations on the circle defined by L
	"""

	B = L*e
	Bsq = (B*B)()
	print 'L = X^Y^e is a non-euclidian line'
	print 'B = L*e =',B
	print 'B^2 =',Bsq
	print 'L^2 =',(L*L)()

	#make_scalars('s c Binv')
	make_symbols('s c Binv M S C alpha')
	"""
	s = sinh(alpha/2)
	c = cosh(alpha/2)
	Binv = 1/sqrt(B*B)  It can be shown that B*B > 0
	"""

	Bhat = Binv*B # Normalize translation generator
	R = c+s*Bhat # Rotor R = exp(alpha*Bhat/2)
	print 's = sinh(alpha/2) and c = cosh(alpha/2)'
	print 'R = exp(alpha*B/(2*|B|)) =',R
	Z = R*X*R.rev()
	Z.expand()
	Z.collect([Binv,s,c,XdotY])
	print 'R*X*R.rev() =',Z
	#W = Z^Y
	W = Z|Y
	W.expand()
	W.collect([s*Binv])
	print '(R*X*rev(R)).Y =',W
	M = 1/Bsq
	W.subs(Binv**2,M)
	W.simplify()
	Bmag = sympy.sqrt(XdotY**2-2*XdotY*Xdote*Ydote)
	W.collect([Binv*c*s,XdotY])

	"""
	Coefficients of W blades must be reduced via hyperbolic trig
	substitutions to yield solution for alpha.
	"""
	W.subs(2*XdotY**2-4*XdotY*Xdote*Ydote,2/(Binv**2))
	W.subs(2*c*s,S)
	W.subs(c**2,(C+1)/2)
	W.subs(s**2,(C-1)/2)
	W.simplify()
	W.subs(1/Binv,Bmag)
	W = W()
	print '(R*X*R.rev()).Y =',W
	nl = '\n'
	Wd = collect(W,[C,S],evaluate=False)
	lhs = Wd[ONE]+Wd[C]*C
	rhs = -Wd[S]*S
	lhs = lhs**2
	rhs = rhs**2
	W = (lhs-rhs).expand()
	W = (W.subs(S**2,C**2-1)).expand()
	W = collect(W,[C**2,C],evaluate=False)
	a = W[C**2]
	b = W[abs(C)]
	c = W[ONE]
	D = (b**2-4*a*c).expand()
	print 'Setting to 0 and solving for C gives:'
	print 'Descriminant D = b^2-4*a*c =',D
	C = (-b/(2*a)).expand()
	print 'C = cosh(alpha) = -b/(2*a) =',C
\end{verbatim}

Note that {\tt Binv} is $\abs{B}^{-1}$.  The critical outputs are the
expressions for {\tt B\verb!^!2} and {\tt (R*X*R.rev())\verb!|Y!}. Then we solve
for {\tt (R*X*R.rev())\verb!|!Y = 0}.  Most of the code is reducing {\tt (R*X*R.rev())\verb!|Y! = 0}
to a quadratic equation in $\cosh\lp\alpha\rp$ which we can solve.  The output of the code is:
\begin{verbatim}
xample: non-euclidian distance calculation
L = X^Y^e is a non-euclidian line
B = L*e = X^Y
+{-(Y.e)}X^e
+{(X.e)}Y^e

B^2 = (X.Y)**2 - 2*(X.Y)*(X.e)*(Y.e)
L^2 = (X.Y)**2 - 2*(X.Y)*(X.e)*(Y.e)
s = sinh(alpha/2) and c = cosh(alpha/2)
R = exp(alpha*B/(2*|B|)) = {c}1
+{Binv*s}X^Y
+{-(Y.e)*Binv*s}X^e
+{(X.e)*Binv*s}Y^e

R*X*R.rev() = {c**2 + Binv*(2*(X.Y)*c*s - 2*(X.e)*(Y.e)*c*s) 
               + Binv**2*((X.Y)**2*s**2 - 2*(X.Y)*(X.e)*(Y.e)*s**2)}X
              +{2*Binv*c*s*(X.e)**2}Y
               +{Binv**2*(-2*(X.e)*(X.Y)**2*s**2 + 4*(X.Y)*(Y.e)*(X.e)**2*s**2) 
               - 2*(X.Y)*(X.e)*Binv*c*s}e

(R*X*rev(R)).Y = {(X.Y)*c**2 + Binv*s*(2*c*(X.Y)**2 - 4*(X.Y)*(X.e)*(Y.e)*c) 
                   + Binv**2*s**2*((X.Y)**3 - 4*(X.e)*(Y.e)*(X.Y)**2 
                   + 4*(X.Y)*(X.e)**2*(Y.e)**2)}1

(R*X*R.rev()).Y = (X.Y)*C+(X.e)*(Y.e)+S*((X.Y)**2-2*(X.Y)*(X.e)*(Y.e))**(1/2)-(X.e)*(Y.e)*C
Setting to 0 and solving for C gives:
Descriminant D = b^2-4*a*c = 0
C = cosh(alpha) = -b/(2*a) = 1 - (X.Y)/(X.e)/(Y.e)
\end{verbatim}
Due to the way sympy processes the output symbolic manipulations {\tt 1 - (X.Y)/(X.e)/(Y.e) =
1 - (X.Y)/((X.e)*(Y.e))}.  Using the value calculated for {\tt B\verb!^!2} 
(the quantity in the radical is always positive for any point 
on the Poincare disk) we have
\be 
\abs{B} = \sqrt{\lp X\cdot Y\rp^{2}-2\lp X\cdot Y\rp\lp X\cdot e\rp\lp Y\cdot e \rp}
\ee
The equation for {\tt X\verb!|!Y = 0} is then
\be\label{eq9}
\hspace{-0.25in}\lp X\cdot Y- \lp X\cdot e\rp\lp Y\cdot e \rp \rp\cosh\lp\alpha\rp+\lp X\cdot e\rp\lp Y\cdot e \rp+
\lp X\cdot Y \rp\lp X\cdot e\rp\lp Y\cdot e \rp\abs{B}\sinh\lp\alpha \rp = 0
\ee
Equation\ref{eq9} is identical to 10.161 in Doran and
Lazenby.  If one employs the standard hyperbolic trig identities we get a quadratic 
equation in $\cosh\lp\alpha\rp$ with solution 
\be
	\cosh\lp\alpha\rp = 1 - \bfrac{X\cdot Y}{\lp X\cdot e\rp\lp Y\cdot e \rp}
\ee
which is the equation in Doran and 
Lazenby without the intermediate geometric algebra manipulations in their equations~10.156 through 10.161.

The biggest problem in using GAsym in this case is having enough understanding of sympy to be able to 
simplify the symbolic coefficient expansions so that we can solve for $\cosh\lp\alpha\rp$. The easiest thing to do
 was to use the metric tensor to enforce the required auxiliary conditions on $X$, $Y$, and $e$. 

\subsection{Conformal Geometry}

We shall show that blades in conformal space represent basis geometric shapes such as circles, lines, spheres,
and planes. In chapter~10 of {\bf Geometric Algebra for Physicists} by Doran and Lazenby it is shown that points
in euclidian space can be represented by rays (null vectors) in a space with two appended dimensions. If basis
vectors for the appended dimensions are $e$ and $\ebar$ where $e^{2} = -\ebar^2 = 1$ we define the null basis
vectors $n = e+\ebar$ and $\nbar = e-\ebar$ and have from the definitions that $n^{2} = \nbar = 0$ and
$n\cdot nbar = 2$. A vector $x$ in the euclidian space is mapped into a ray in the conformal space by
\be\label{eq12}
X = \Fof{x} = \half\lp x^{2}n+2x-\nbar\rp
\ee
Now let $A$, $B$, $C$, and $D$ be rays representing fixed points in a 3-D euclidian space and $X$ representing
the variable point $x = x_{1}e_{1}+x_{2}e_{2}+x_{3}e_{3}$ via the mapping in equation~\ref{eq12}.  Then the
basis geometric shapes are represented by the blade equations:

\begin{center}
\begin{tabular}{ccc}
Line & - & $\lp A\W B\W n \rp\W X = 0$ \\
Circle & - & $\lp A\W B\W C \rp\W X = 0$ \\
Plane & - & $\lp A\W B\W C\W n \rp\W X = 0$ \\
Sphere & - & $\lp A\W B\W C\W D \rp\W X = 0$ \\
\end{tabular}
\end{center}

The python code demonstrating this is shown below.  The points on the circle and sphere are selected so that both
shapes have centers at the origin and the form of the equation in $x_{1}$, $x_{2}$, and $x_{3}$ are obvious.

\begin{verbatim}
def F(x):
    Fx = HALF*((x*x)*n+2*x-nbar)
    return(Fx)

def make_vector(a,n = 3):
    if type(a) == types.StringType:
        sym_str = ''
        for i in range(n):
        sym_str += a+str(i)+' '
        sym_lst = make_symbols(sym_str)
        sym_lst.append(ZERO)
        sym_lst.append(ZERO)
        a = MV(sym_lst,'vector')
    return(F(a))
\end{verbatim}

\begin{verbatim}
    print '\n\n\nExample: Conformal representations of circles, lines,\n'+\
          ' spheres, and planes'

    metric = '1 0 0 0 0,'+ \
             '0 1 0 0 0,'+ \
             '0 0 1 0 0,'+ \
             '0 0 0 0 2,'+ \
             '0 0 0 2 0'

    MV.setup('e0 e1 e2 n nbar',metric,debug=0)
    MV.set_str_format(1)

    #conformal representation of points
    A = make_vector(e0)    # point a = (1,0,0)  A = F(a)
    B = make_vector(e1)    # point b = (0,1,0)  B = F(b)
    C = make_vector(-1*e0) # point c = (-1,0,0) C = F(c)
    D = make_vector(e2)    # point d = (0,0,1)  D = F(d)
    X = make_vector('x')
    print 'a = e0, b = e1, c = -e0, and d = e2'
    print 'A = F(a) = 1/2*(a*a*n+2*a-nbar), etc.'
    print 'Circle through a, b, and c'
    print 'Circle: A^B^C^X = 0 =',(A^B^C^X)
    print 'Line through a and b'
    print 'Line  : A^B^n^X = 0 =',(A^B^n^X)
    print 'Sphere through a, b, c, and d'
    print 'Sphere: A^B^C^D^X = 0 =',(A^B^C^D^X)
    print 'Plane through a, b, and d'
    print 'Plane : A^B^n^D^X = 0 =',(A^B^n^D^X)
\end{verbatim}

The output of the code is shown below.  This output show that the blade
equations properly encode the basic geometric shapes.  The shapes shown
can be translated and scaled using rotations in the conformal space. Thus
the blade equations are corrent for any line, circle, plane, or sphere. 

\begin{verbatim}
Example: Conformal representations of circles, lines, spheres, and planes
a = e0, b = e1, c = -e0, and d = e2
A = F(a) = 1/2*(a*a*n+2*a-nbar), etc.
Circle through a, b, and c
Circle: A^B^C^X = 0 = {-x2}e0^e1^e2^n
+{x2}e0^e1^e2^nbar
+{-1/2+1/2*x0**2+1/2*x2**2+1/2*x1**2}e0^e1^n^nbar

Line through a and b
Line  : A^B^n^X = 0 = {-x2}e0^e1^e2^n
+{-1/2+1/2*x1+1/2*x0}e0^e1^n^nbar
+{1/2*x2}e0^e2^n^nbar
+{-1/2*x2}e1^e2^n^nbar

Sphere through a, b, c, and d
Sphere: A^B^C^D^X = 0 = {1/2-1/2*x0**2-1/2*x2**2-1/2*x1**2}e0^e1^e2^n^nbar

Plane through a, b, and d
Plane : A^B^n^D^X = 0 = {1/2-1/2*x1-1/2*x0-1/2*x2}e0^e1^e2^n^nbar
\end{verbatim}

\subsection{Reciprocal Frames}

We shall show that the set of reciprocal vectors with respect to an arbitrary basis are
correctly calculated according to {\bf Geometric Algebra for Physicists} by Doran and Lazenby,
chapter~4, page~100. The only simplification we make is that our basis consists of unit
vectors $e_{1}^{2} = e_{2}^{2} = e_{3}^{2} = 1$, otherwise $e_{1}$, $e_{2}$, and $e_{3}$
are arbitrary. We program formula~4.94, the program implementing this is shown below. In
the program we have $e^{j} = {\tt Ej}$ and $E_{n} = {\tt E}$ for formulas~4.92 and 4.94 in
the reference. 
\begin{verbatim}
    metric = '1 # #,'+ \
             '# 1 #,'+ \
             '# # 1,'

    MV.setup('e1 e2 e3',metric)
    print 'Example: Reciprocal Frames e1, e2, and e3 unit vectors.\n\n'
    E = e1^e2^e3
    Esq = (E*E)()
    print 'E =',E
    print 'E^2 =',Esq
    Esq_inv = 1/Esq
    E1 = (e2^e3)*E
    E2 = (-1)*(e1^e3)*E
    E3 = (e1^e2)*E
    print 'E1 = (e2^e3)*E =',E1
    print 'E2 =-(e1^e3)*E =',E2
    print 'E3 = (e1^e2)*E =',E3
    w = (E1|e2)
    w.collect(MV.g)
    print 'E1|e2 =',w
    w = (E1|e3)
    w.collect(MV.g)
    print 'E1|e3 =',w
    w = (E2|e1)
    w.collect(MV.g)
    print 'E2|e1 =',w
    w = (E2|e3)
    w.collect(MV.g)
    print 'E2|e3 =',w
    w = (E3|e1)
    w.collect(MV.g)
    print 'E3|e1 =',w
    w = (E3|e2)
    w.collect(MV.g)
    print 'E3|e2 =',w
    w = (E1|e1)
    w = expand(w())
    Esq = expand(Esq)
    print '(E1|e1)/E^2 =',w/Esq
    w = (E2|e2)
    w = expand(w())
    print '(E2|e2)/E^2 =',w/Esq
    w = (E3|e3)
    w = expand(w())
    print '(E3|e3)/E^2 =',w/Esq
\end{verbatim}
The output of the program is below and shows that $e^{i}\cdot e_{j} = \delta^{i}_{j}$,
where $\delta^{i}_{j}$ is the Knonecker delta function (one if $i = j$, zero if $i \ne j$).  
The output is correct
\begin{verbatim}
Example: Reciprocal Frames e1, e2, and e3 unit vectors.
E = e1^e2^e3
E^2 = -1+(e1.e3)**2+(e2.e3)**2-2*(e1.e2)*(e2.e3)*(e1.e3)+(e1.e2)**2
E1 = (e2^e3)*E = {-1+(e2.e3)**2}e1
                +{(e1.e2)-(e2.e3)*(e1.e3)}e2
                +{-(e1.e2)*(e2.e3)+(e1.e3)}e3
E2 =-(e1^e3)*E = {(e1.e2)-(e2.e3)*(e1.e3)}e1
                +{-1+(e1.e3)**2}e2
                +{-(e1.e2)*(e1.e3)+(e2.e3)}e3
E3 = (e1^e2)*E = {-(e1.e2)*(e2.e3)+(e1.e3)}e1
                +{-(e1.e2)*(e1.e3)+(e2.e3)}e2
                +{-1+(e1.e2)**2}e3
E1|e2 = 0
E1|e3 = 0
E2|e1 = 0
E2|e3 = 0
E3|e1 = 0
E3|e2 = 0
(E1|e1)/E^2 = 1
(E2|e2)/E^2 = 1
(E3|e3)/E^2 = 1
\end{verbatim}
\subsection{More Examples}
The complete test suite ({\tt test\_symbolicGA.py}) and its output ({\tt test\_symbolicGA.out}) are part of the
GAsympy documentation and include more examples than shown in this document.  Documentation of all the classes
and functions in GAsympy are obtained by the command line {\tt pydoc GAsympy} for the generation of a text 
file or {\tt pydoc -w GAsympy} for the generation of a html file documenting GAsympy in the GAsympy directory.

\section{GAcalc.py - A Calculator for Geometric Algebra}
If one is not familiar with python programming a simple Geometric Algebra calculator is included with symbolicGA.  To run GAcalc.py use the command line {\tt GAcalc.py [savefile]}.  {\tt savefile} is an optional command line argument that
allows one to save or restore their calculator session.  If the file {\tt savefile} does not exist GAcalc 
creates it and saves the
current GAcalc session in it.  If {\tt savefile} exists, GAcalc reads it and executes the commands saved from a previous
session restoring and making available all the quantities calculated in the previous session to the current session. Note 
one should never use a {\tt savefile} name that ends in {\tt .py} since this causes problems in the python interpeture
that will cause the calculator to lock up.  An example of a GAcalc session follows:

\begin{verbatim}
$ GAcalc.py session.sav
In: make_symbols('x0 x1 x2')
Out: [x0 x1 x2]
In: x = x0*a0+x1*a1+x2*a2
Out: x = {x0}a0+{x1}a1+{x2}a2
In: y = a0^a1^a2
Out: y = a0^a1^a2
In: x^y
Out: 0
In: y = a0^a1
Out: y = a0^a1
In: x^y
Out: {x2}a0^a1^a2
In: z = x+y
Out: z = {x0}a0+{x1}a1+{x2}a2+a0^a1
In: z = z+(a0^a1^a2)
Out: z = {x0}a0+{x1}a1+{x2}a2+a0^a1+a0^a1^a2
In: z.even()
Out: a0^a1
In: z.odd()
Out: {x0}a0+{x1}a1+{x2}a2+a0^a1^a2
In: z.rev()
Out: {x0}a0+{x1}a1+{x2}a2-a0^a1-a0^a1^a2
In: z = z+1
Out: z = 1+{x0}a0+{x1}a1+{x2}a2+a0^a1+a0^a1^a2
In: ?
\end{verbatim}

The default basis for GAcalc is {\tt 'a0 a1 a2'} with the corresponding default metric. This can be
overidden by inputing {\tt MV.setup(your basis,your metric)}. Also it is absolutely critical to enclose
operations with \verb!^! and {\tt |} in parenthesis since they have a lower priority than {\tt +}, 
{\tt -}, and {\tt *}.

\end{document}
